\section{A Non--Overlapping Domain Decomposition Method}
\label{Sec:DDM}

We now restate our governing equations $(4.8)$ in terms
 of surface quantities via a Non--Overlapping \gls{ddm} \cite{Lions90,DespresPhD,Despres91}. For this we define
\begin{gather*}
U(x) := u(x,g(x)),
\quad
\tilde{U}(x) := -\partial_N u(x,g(x)),
\\
W(x) := w(x,g(x)),
\quad
\tilde{W}(x) := \partial_N w(x,g(x)),
\end{gather*}
where $u$ is a $d$--periodic solution of $(4.8\text{a})$ and
$(4.8\text{e})$, and $w$ is a $d$--periodic solution of
$(4.8\text{b})$ and $(4.8\text{f})$. In terms of these
our full governing equations $(4.8)$ are equivalent to the
pair of boundary conditions, $(4.8\text{c})$ \&
$(4.8\text{d})$,
\be
U - W = \zeta,
\quad
-\tilde{U} - (i \alpha) (\partial_x g) U 
  - \tau^2 \left[ \tilde{W} - (i \alpha) (\partial_x g) W \right]
  = \psi.
\ee
This set of two equations for four unknowns can be closed by noting that
the pairs $\{ U, \tilde{U} \}$ and $\{ W, \tilde{W} \}$ are connected,
e.g., by the DNOs
\bes
G: U \rightarrow \tilde{U},
\quad
J: W \rightarrow \tilde{W}.
\ees
\begin{definition} We recall the precise definition of the upper layer DNO
\cite{Nicholls16b}:
Given an integer $s \geq 0$, if 
$g \in C^{s+2}$ the unique solution of

\begin{subequations}
\label{Eqn:Helm:Upper}
\begin{align}
& \Delta u + 2 i \alpha \partial_x u + (\gamma^u)^2 u = 0,
  && g(x) < z < a, \\
& u(x,g(x)) = U(x), 
  && z = g(x), \\
& \partial_z u(x,a) - T^u[ u(x,a) ] = 0,
  && z = a, \\
& u(x+d,z)=u(x,z),
\end{align}
\end{subequations}
defines the Upper Layer DNO
\be
\label{Eqn:G}
G(g): U \rightarrow \tilde{U} := -(\partial_N u)(x,g(x)).
\ee
\end{definition}


\begin{definition} Similarly, we recall the definition of the lower layer DNO:
Given an integer $s \geq 0$, if 
$g \in C^{s+2}$ the unique solution of
\begin{subequations}
\label{Eqn:Helm:Lower}
\begin{align}
& \Delta w + 2 i \alpha \partial_x w + (\gamma^w)^2 w = 0,
  && -b < z < g(x), \\
& w(x,g(x)) = W(x), 
  && z = g(x), \\
& \partial_z w(x,-b) - T^w[ w(x,-b) ] = 0,
  && z = -b,\\
& w(x+d,z)=w(x,z),
\end{align}
\end{subequations}
defines the Lower Layer DNO
\be
\label{Eqn:J}
J(g): W \rightarrow \tilde{W} := (\partial_N w)(x,g(x)).
\ee
\end{definition}
\begin{flushleft}
We now write $(4.9)$ as
\end{flushleft}
\be
\label{Eqn:AVR}
\bA \bV = \bR,
\ee
where
\be
\label{Eqn:AVR:Def}
\bA = \begin{pmatrix} I & -I \\ 
  G + (\partial_x g) (i \alpha) & \tau^2 J - \tau^2 (\partial_x g) (i \alpha)
  \end{pmatrix},
\quad
\bV = \begin{pmatrix} U \\ W \end{pmatrix},
\quad
\bR = \begin{pmatrix} \zeta \\ -\psi \end{pmatrix}.
\ee
For later use, the trivial flat--interface version of $(4.15)$ is $\bA_{0,0}\bV_{0,0}=\bR_{0,0}$ where
\be
\label{Eqn:AVR_Flat:Def}
\bA_{0,0} = \begin{pmatrix}I & -I\\
-G_{0,0} & -\tau^2J_{0,0}\end{pmatrix},
\quad
\bV_{0,0} = \begin{pmatrix} U_{0,0} \\ W_{0,0} \end{pmatrix},
\quad
\bR_{0,0} = \begin{pmatrix} \zeta_{0,0} \\ -\psi_{0,0} \end{pmatrix}.
\ee