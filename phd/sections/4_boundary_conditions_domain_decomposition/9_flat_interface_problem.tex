\section{The Flat--Interface Problem}
\label{Sec:Flat Interface}
As we outlined in Theorem $4.6.1$, the key to our developments (as with all regular perturbation arguments) is the flat--interface version of $(4.14)$
\bes
\bA_{0,0}\bV_{0,0}=\bR_{0,0},
\ees
in particular the invertibility of $\bA_{0,0}$ and the mapping properties of $\bA_{0,0}^{-1}$. From $(4.15)$, it is not hard to see that the formulas for $\bA$ and $\bR$ are 
\begin{subequations}
\label{Eqn:An:Rn}
\begin{align}
\bA_{0,0} & = \begin{pmatrix} I & -I \\ G_{0,0} & \tau^2 J_{0,0} \end{pmatrix}, \\
\bA_{n,m} & = \begin{pmatrix} 0 & 0 \\ G_{n,m} & \tau^2 J_{n,m} \end{pmatrix} 
  \notag \\
  & \quad + \delta_{n,1} \left\{ 1 + \delta_{m,1} \right\} (\partial_x f) (i \ualpha)  
  \begin{pmatrix} 0 & 0 \\ 1 & -\tau^2 \end{pmatrix},
  && \text{$n \neq 0$ or $m \neq 0$}, \\
\bR_{n,m} & = \begin{pmatrix} \zeta_{n,m} \\ -\psi_{n,m} \end{pmatrix},
\end{align}
\end{subequations}
where $\delta_{n,m}$ is the Kronecker delta function. We note that $\bA_{0,0}$ is diagonalized by
the Fourier transform so that $\bA_{0,0} \bV_{n,m} = \bR_{n,m}$
can be expressed as

\bes
\sump \widehat{\bA}_{0,0}(p) \widehat{\bV}_{n,m}(p) e^{i \tilde{p} x}
  = \sump \widehat{\bR}_{n,m}(p) e^{i \tilde{p} x},
\ees
which implies
\bes
\widehat{\bV}_{n,m}(p) = \left[ \widehat{\bA}_{0,0}(p) \right]^{-1}
  \widehat{\bR}_{n,m}(p).
\ees
It is not difficult to see 
\bes
\widehat{\bA}_{0,0}(p) = \begin{pmatrix} 1 & -1 \\
  (-i \gamma^u_p) & \tau^2 (-i \gamma^w_p) \end{pmatrix},
\ees
cf.\ $(4.29)$, implying 
\bes
\left[ \widehat{\bA}_{0,0}(p) \right]^{-1} 
  = \frac{1}{\hat{\Delta}_p}
  \begin{pmatrix} \tau^2 (-i \gamma^w_p) & 1 \\
  (i \gamma^u_p) & 1 \end{pmatrix},
\quad
\hat{\Delta}_p := -(i \gamma^u_p + \tau^2 (i \gamma^w_p)).
\ees
\begin{remark}
From these formulas it becomes obvious that the operator
$\bA_{0,0}$ is always invertible and our algorithm is 
well--defined. Recalling that we assume
a dielectric in the upper layer (so that the incident radiation
propagates) we have that $\gamma^u_p$ is either real and positive
or purely imaginary (with positive imaginary part). If a
dielectric fills the lower layer then we have the same state of
affairs for $\gamma^w_p$ so that, given that $\tau^2$ will be
positive and real, $\Delta_p \neq 0$. Alternatively, if a metal
fills the lower layer then $\gamma^w_p$ will be complex with
positive imaginary part. While it is less obvious, this ensures
that, once again, $\Delta_p \neq 0$.
\end{remark}

We now verify Item $3$ in Theorem $4.5.1$. By the analysis above, we know that
\be
\bA_{0,0} = \begin{pmatrix} I & -I \\ G_{0,0} & \tau^2 J_{0,0} \end{pmatrix},
\ee
where
\be
G_{0,0} = -i \gamma^u_D,
\quad
J_{0,0} = -i \gamma^w_D,
\ee
are order--one Fourier multipliers defined by
\be
G_{0,0}[U] = \sump (-i \gamma^u_p) \hat{U}_p e^{i \tilde{p} x},
\quad
J_{0,0}[W] = \sump (-i \gamma^w_p) \hat{W}_p e^{i \tilde{p} x}.
\ee

\begin{lemma}
The linear operator $\bA_{0,0}$ maps $X^s$ to $Y^s$ boundedly, is invertible, and its inverse maps $Y^s$ to $X^s$ boundedly.
\end{lemma}
\begin{proof}{[Lemma 4.8.1]}
We begin by defining the operator
\bes
\Delta := G_{0,0} + \tau^2 J_{0,0} 
  = (-i \gamma^u_D) + \tau^2 (-i \gamma^w_D),
\ees
which has Fourier symbol
\bes
\hat{\Delta}_p = (-i \gamma^u_p) + \tau^2 (-i \gamma^w_p),
\ees
and noting that there exist positive constants
$C_G$, $C_J$, and $C_{\Delta}$ such that
\bes
\Abs{-i \gamma^u_p} \leq C_G \Angle{\tilde{p}},
\quad
\Abs{-i \gamma^w_p} \leq C_J \Angle{\tilde{p}},
\quad
\Abs{\hat{\Delta}_p} \leq C_{\Delta} \Angle{\tilde{p}}.
\ees
Importantly, provided that $n^u \neq n^w$, it is
not difficult to establish the crucial fact that $\hat{\Delta}_p \neq 0$. Finally, one
can also find a positive constant $C_{\Delta^{-1}}$ such that
\bes
\Abs{\frac{1}{\hat{\Delta}_p}} \leq 
  C_{\Delta^{-1}} \Angle{\tilde{p}}^{-1}.
\ees
With this it is a simple matter to realize that $\Delta^{-1}$ exists
and that
\bes
\Delta: H^{s+3/2} \rightarrow H^{s+1/2},
\quad
\Delta^{-1}: H^{s+1/2} \rightarrow H^{s+3/2}.
\ees
Next, we write generic elements of $X^s$ and $Y^s$ as
\bes
\bV = \begin{pmatrix} U \\ W \end{pmatrix} \in X^s,
\quad
\bR = \begin{pmatrix} \zeta \\ -\psi \end{pmatrix} \in Y^s.
\ees
Using the definitions of the norms of $X^s$ and $Y^s$, 
and the facts
\bes
2 a b \leq a^2 + b^2,
\quad
\Norm{A+B}{}^2 \leq (\Norm{A}{} + \Norm{B}{})^2,
\ees
we find that
\begin{align*}
\Norm{\bA_{0,0} \bV}{Y^s}^2
  & = \SobNorm{U - W}{s+3/2}^2 
    + \SobNorm{G_{0,0} U + \tau^2 J_{0,0} W}{s+1/2}^2 \\
  & \leq 2 \SobNorm{U}{s+3/2}^2 
    + 2 \SobNorm{W}{s+3/2}^2
    + C_G^2 \SobNorm{U}{s+3/2}^2 \\
  & \quad
    + \tau^2 C_G C_J \left(\SobNorm{U}{s+3/2}^2 + \SobNorm{W}{s+3/2}^2 \right)
    + C_J^2 \tau^4 \SobNorm{W}{s+3/2}^2 \\
  & \leq \max \{ 2, C_G^2, \tau^2 C_G C_J, \tau^4 C_J^2 \} 
    \left( \SobNorm{U}{s+3/2}^2 + \SobNorm{W}{s+3/2}^2 \right) \\
  & = \max \{ 2, C_G^2, \tau^2 C_G C_J, \tau^4 C_J^2 \}
  \Norm{\bV}{X^s}^2,
\end{align*}
so that $\bA_{0,0}$ does indeed map $X^s$ to $Y^s$ boundedly.
We define the operator
\bes
\bB := \Delta^{-1} \begin{pmatrix} \tau^2 J_{0,0} & I \\
  -G_{0,0} & I \end{pmatrix},
\ees
and note that
\bes
\bB \bA_{0,0} = \bA_{0,0} \bB 
  = \begin{pmatrix} I & 0 \\ 0 & I \end{pmatrix},
\ees
so that the inverse of $\bA_{0,0}$ exists and 
$\bA_{0,0}^{-1} = \bB$. Furthermore, as above,
\begin{align*}
\Norm{\bA_{0,0}^{-1} \bR}{X^s}^2
  & = \SobNorm{\Delta^{-1} (\tau^2 J_{0,0} \zeta - \psi)}{s+3/2}^2 
    + \SobNorm{\Delta^{-1} (-G_{0,0} \zeta - \psi)}{s+3/2}^2 \\
  & \leq C_{\Delta^{-1}}^2 \tau^4 C_J^2 \SobNorm{\zeta}{s+3/2}^2 
    + C_{\Delta^{-1}}^2 \tau^2 C_J 
    \left( \SobNorm{\zeta}{s+3/2}^2 + 
    \SobNorm{\psi}{s+1/2}^2 \right) \\
  & \quad 
    + C_{\Delta^{-1}}^2 C_G^2 \SobNorm{\zeta}{s+3/2}^2 
    + C_{\Delta^{-1}}^2 C_G \left( \SobNorm{\zeta}{s+3/2}^2 + 
    \SobNorm{\psi}{s+1/2}^2 \right) \\
  & \quad  + 2 C_{\Delta^{-1}}^2 \SobNorm{\psi}{s+1/2}^2  \\
  & \leq C_{\Delta^{-1}}^2 \max \{ 2, C_G, C_G^2,
    \tau^2 C_J, \tau^4 C_J^2 \}
    \left( \SobNorm{\zeta}{s+3/2}^2 + \SobNorm{\psi}{s+1/2}^2 \right) \\
  & = C_{\Delta^{-1}}^2 \max \{ 2, C_G, C_G^2,
    \tau^2 C_J, \tau^4 C_J^2 \}
    \Norm{\bR}{Y^s}^2,
\end{align*}
and $\bA_{0,0}^{-1}$ maps $Y^s$ to $X^s$ boundedly.
\end{proof}