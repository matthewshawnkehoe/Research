\section{Governing Equations and Propagating Conditions}
\label{intro:governing equations}
The two--layer scattering problem is composed of outgoing, quasiperiodic solutions to
\begin{subequations}
\label{Eqn:Govern_With_Phase}
\begin{align}
& \Delta \tilde{u} + (k^u)^2 \tilde{u} = 0,&& z > g(x),
  \label{Eqn:Govern_With_Phase:a} \\
& \Delta \tilde{w} + (k^w)^2 \tilde{w} = 0,&& z < g(x),
  \label{Eqn:Govern_With_Phase:b} \\
& \tilde{u} - \tilde{w} = \tilde{\zeta}, && \text{at $z = g(x)$},
  \label{Eqn:Govern_With_Phase:c} \\
& \partial_N \tilde{u}-\tau^2\partial_N \tilde{w}=\tilde{\psi}, &&\text{at $z=g(x)$.}
  \label{Eqn:Govern_With_Phase:d}
\end{align}
The Dirichlet and Neumann data are

\begin{align}
\tilde{\zeta}(x) & := -e^{i\alpha x -i \gamma^u g(x)}, 
  \label{Eqn:Govern_With_Phase:e} \\
\tilde{\psi}(x) & := (i \gamma^u + i \alpha (\partial_x g)) e^{i\alpha x -i \gamma^u g(x)},
  \label{Eqn:Govern_With_Phase:f}
\end{align}
\end{subequations}
and
\bes
\tau^2 = \begin{cases} 1, & \text{TE}, \\
  (k^u/k^w)^2 = (n^u/n^w)^2, & \text{TM}.
  \end{cases}
\ees
Following our analysis in Chapters 2 and 3, we start by removing the phase in $(4.1)$ through the relationship $v(x,z)=e^{-i\alpha x}\tilde{v}(x,z)$, $v\in\{u,w\}$, and $\zeta(x)=e^{-i\alpha x}\tilde{\zeta}(x)$, $\psi(x)=e^{-i\alpha x}\tilde{\psi}(x)$. This yields outgoing, $d$--periodic solutions of
\begin{subequations}
\label{Eqn:Govern}
\begin{align}
& \Delta u + 2 i \alpha \partial_x u + (\gamma^u)^2 u = 0,&& z > g(x),
  \label{Eqn:Govern:a} \\
& \Delta w + 2 i \alpha \partial_x w + (\gamma^w)^2 w = 0,&& z < g(x),
  \label{Eqn:Govern:b} \\
& u - w = \zeta, && \text{at $z = g(x)$},
  \label{Eqn:Govern:c} \\
& \partial_N u - i \alpha (\partial_x g) u
  - \tau^2 \left[ \partial_N w - i \alpha (\partial_x g) w \right]
  = \psi, && \text{at $z = g(x)$},
  \label{Eqn:Govern:d}
\end{align}
where 
\begin{align}
\zeta(x) & := -e^{-i \gamma^u g(x)}, 
  \label{Eqn:Govern:e} \\
\psi(x) & := (i \gamma^u + i \alpha (\partial_x g)) e^{-i \gamma^u g(x)},
  \label{Eqn:Govern:f}
\end{align}
\end{subequations}
and the left--hand side of $(4.2\text{d})$ follows from
\begin{align*}
\partial_N \tilde{u}-\tau^2\partial_N \tilde{w}&=\partial_N \left(e^{i\alpha x}u\right) - \tau^2\partial_N\left(e^{i\alpha x}w\right) \\&=
e^{i\alpha x}\Big(\partial_z u + (-\partial_xg)\partial_xu-(i\alpha)(\partial_xg)u ~-\\& \qquad \qquad\tau^2\Big[\partial_zw + (-\partial_xg)\partial_xw - (i\alpha)(\partial_xg)w\Big]\Big)\\&=
e^{i\alpha x}\Big(\partial_N u - i \alpha (\partial_x g) u
  - \tau^2 \left[ \partial_N w - i \alpha (\partial_x g) w \right]\Big).
\end{align*}
The \gls{upc} and \gls{dpc}
\cite{ArensHab} rigorously enforce the Outgoing Wave Conditions which we mentioned in $\S 1.8$. We now demonstrate how these can be stated in terms of Transparent
Boundary Conditions which also truncate the bi--infinite problem domain to one
of finite size. As discussed in $\S 1.8$, we choose values $a$ and $b$ such that
\bes
a > \SupNorm{g},
\quad
-b < -\SupNorm{g},
\ees
and define the artificial boundaries $\{ z = a \}$ and $\{ z = -b \}$. In 
$\{ z > a \}$ the Rayleigh expansions \cite{Petit80} tell us that upward propagating
solutions of $(4.2\text{a})$ are
\be
\label{Eqn:Rayleigh:u}
u(x,z) = \sump a_p e^{i \tilde{p} x + i \gamma^u_p z},
\ee
where, for $p\in\mathbb Z$ and $q \in \{ u, w \}$,
\be
\label{Eqn:p:alphap:gammap:Def}
\tilde{p} := \frac{2 \pi p}{d},
\quad
\alpha_p := \alpha + \tilde{p},
\quad
\gamma^q_p := \sqrt{ (k^q)^2 - \alpha_p^2 },
\quad
\text{Im} \left\{ \gamma^q_p \right\} \geq 0.
\ee
In a similar fashion, downward propagating solutions of $(4.2\text{b})$ in
$\{ z < -b \}$ can be expressed as
\be
w(x,z) = \sump d_p e^{i \tilde{p} x - i \gamma^w_p z}.
\ee
With these we can define the Transparent Boundary Conditions in the following way:
Focusing on the UPC we rewrite $(4.3)$ as
\bes
u(x,z) = \sump \left( a_p e^{i \gamma^u_p a} \right) e^{i \tilde{p} x + i \gamma^u_p (z-a)}
  = \sump \hat{\xi}_p e^{i \tilde{p} x + i \gamma^u_p (z-a)},
\ees
and note that,
\bes
u(x,a) = \sump \hat{\xi}_p e^{i \tilde{p} x} =: \xi(x),
\ees
and
\bes
\partial_z u(x,a) = \sump (i \gamma^u_p) \hat{\xi}_p e^{i \tilde{p} x} =: T^u[\xi(x)],
\ees
which defines the order--one Fourier multiplier $T^u$. For the DPC we rewrite $(4.5)$ as
\bes
w(x,z) = \sump \left( d_p e^{i \gamma^w_p b} \right) e^{i \tilde{p} x - i \gamma^w_p (z+b)}
  = \sump \hat{\psi}_p e^{i \tilde{p} x - i \gamma^w_p (z+b)},
\ees
and keep in mind,
\bes
w(x,-b) = \sump \hat{\psi}_p e^{i \tilde{p} x} =: \psi(x),
\ees
and
\bes
\partial_z w(x,-b) = \sump (-i \gamma^w_p) \hat{\psi}_p e^{i \tilde{p} x} =: T^w[\psi(x)],
\ees
which defines the order--one Fourier multiplier $T^w$. From this we state that
upward--propagating solutions of $(4.2\text{a})$ satisfy the Transparent Boundary
Condition at $z = a$
\be
\label{Eqn:TransBC:u}
\partial_z u(x,a) - T^u[u(x,a)] = 0,
\quad
z = a.
\ee
Similarly, downward--propagating solutions of $(4.2\text{b})$ satisfy the Transparent Boundary
Condition at $z = -b$
\be
\label{Eqn:TransBC:w}
\partial_z w(x,-b) - T^w[w(x,-b)] = 0,
\quad
z = -b.
\ee
We also point out that solutions which satisfy $(4.6)$ and 
$(4.7)$ equivalently satisfy the UPC and DPC, respectively
\cite{ArensHab}. With these we now state the full set of governing equations as
\begin{subequations}
\label{Eqn:Govern:Full}
\begin{align}
& \Delta u + 2 i \alpha \partial_x u + (\gamma^u)^2 u = 0,&& z > g(x), 
  \label{Eqn:Govern:Full:a} \\
& \Delta w + 2 i \alpha \partial_x w + (\gamma^w)^2 w = 0,&& z < g(x), 
  \label{Eqn:Govern:Full:b} \\
& u - w = \zeta, && z = g(x), 
  \label{Eqn:Govern:Full:c} \\
& \partial_N u - i \alpha (\partial_x g) u
  - \tau^2 \left[ \partial_N w - i \alpha (\partial_x g) w \right]
  = \psi, && z = g(x), 
  \label{Eqn:Govern:Full:d} \\
& \partial_z u(x,a) - T^u[u(x,a)] = 0, && z = a, 
  \label{Eqn:Govern:Full:e} \\
& \partial_z w(x,-b) - T^w[w(x,-b)] = 0, && z = -b, 
  \label{Eqn:Govern:Full:f} \\
& u(x+d,z) = u(x,z), \\
& w(x+d,z) = w(x,z).
\end{align}
\end{subequations}

