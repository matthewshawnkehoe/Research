\section{Alternatives to DNOs}
\label{Sec: Alternatives to DNOs}
In Chapter $4$ we wrote our scattering problem as a linear system
\bes
\bA \bV = \bR,
\ees
where, upon expanding $\{\bA,\bV,\bR\}$ in both $\varepsilon$ and $\delta$, we arrived at the flat--interface solution $\bA_{0,0}\bV_{0,0}=\bR_{0,0}$
at order $\mathcal{O}(\varepsilon^0,\delta^0)$. We then saw it was necessary to invert
\bes
\bA_{0,0} = \begin{pmatrix}I & -I\\
-G_{0,0} & -\tau^2J_{0,0}\end{pmatrix}, 
\ees
which features the two DNOs, $G_{0,0}$ and $J_{0,0}$, in order to show the existence and uniqueness of solutions. A primary feature of all HOPS schemes is the inversion of a single, sparse operator $\bA_{0,0}$ through the use of DNOs. However, one may ponder if a different technique could produce a more competitive algorithm that is comparable to our HOPS/AWE algorithm (or even better). Is it absolutely necessary to pass in transformed field data in order to efficiently compute and recover internal information stored at the grating surface?
\\
\newline
\textbf{Predictions:} A primary advantage of our HOPS/AWE scheme is that for every perturbation order, it is only necessary to invert a single sparse operator corresponding to a flat--interface, order--zero approximation. There are a number of competing approaches in general perturbation theory within the context of layered media problems. In regards to electromagnetic wave scattering, Galerkin and boundary element methods are discussed in \cite{escapil2020helmholtz,silva2017quantifying,nakata1990boundary,elschner2012optimization,rathsfeld2006} and a high--order perturbation approach based on boundary integral equations in \cite{dolz2020higher}. High--order schemes for linear waves can be computed using level set methods \cite{sethian1999level} and fast marching methods, as well as other methods involving domain decomposition \cite{el2004comparing,benamou1997domain,larsson1999domain,gong2021convergence,perez2018domain,chan1994domain}. A holistic evaluation of these competing methods could potentially improve our HOPS/AWE algorithm if we found a faster method of inverting linear operators without the use of DNOs.