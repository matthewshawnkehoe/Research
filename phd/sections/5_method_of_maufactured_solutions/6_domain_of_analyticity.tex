\section{The Domain of Analyticity}
\label{Sec:Anal}

While the precise domain of analyticity of our solutions in $(\Eps,\delta)$ cannot be specified, it is clear that the expansion of $\gamma_p^q(\delta)$ only converges for $\delta$ away from the Rayleigh singularities. Therefore, our expansions are only valid on subsets of the $(\Eps,\delta)$--plane away from Rayleigh singularities. For instance, in the upper layer, Rayleigh singularities
occur when $\ualpha_p^2 = (\uk^u)^2$ which implies
\be
\label{Eqn:Rayleigh:Sing}
\uomega = \pm \frac{c_0}{n^u} \left\{ \ualpha + \frac{2 \pi p}{d}
  \right\},
\quad
  \text{for any $p \in \mathbb{Z}$}.
\ee
In the interest of maximizing our choice of $\delta$ we selected
a ``mid--point'' value of $\uomega$ which is as far away as possible
from consecutive Rayleigh singularities
\be
\label{Eqn:uomegaq}
\uomega_q := \frac{c_0}{n^u} \left\{ \ualpha 
  + \frac{2 \pi (q+1/2)}{d} \right\}.
\ee
About this value the nearest singularities are
\begin{align*}
\uomega_q^- & := \frac{c_0}{n^u} \left\{ \ualpha 
  + \frac{2 \pi q}{d} \right\}
  = \uomega_q - \frac{\pi c_0}{n^u d},
\\
\uomega_q^+ & := \frac{c_0}{n^u} \left\{ \ualpha 
  + \frac{2 \pi (q+1)}{d} \right\}
  = \uomega_q + \frac{\pi c_0}{n^u d},
\end{align*}
so to maximize our range of $\omega$ we choose, for some
filling fraction $0 < \sigma < 1$,
\bes
\uomega_q - \sigma \left( \frac{\pi c_0}{n^u d} \right)
  < \omega < \uomega_q + \sigma \left( \frac{\pi c_0}{n^u d} \right).
\ees
To express this in terms of $\delta$ we recall that
$\omega = (1+\delta) \uomega_q$ which gives
\bes
- \sigma \left( \frac{\pi c_0}{ \uomega_q n^u d} \right)
  < \delta < \sigma \left( \frac{\pi c_0}{ \uomega_q n^u d} \right).
\ees
Simplifying gives
\be
\label{Eqn:uomega:Range}
-\left( \frac{\sigma}{(\ualpha d/\pi) + 2 q + 1} \right)
  < \delta < \left( \frac{\sigma}{(\ualpha d/\pi) + 2 q + 1} \right).
\ee