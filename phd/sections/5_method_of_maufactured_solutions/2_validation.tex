\section{The Method of Manufactured Solutions}
\label{intro:validation}

To validate our numerical scheme we utilized the MMS \cite{Burggraf66,Roache02,Roy05}.
To summarize, we considered a general system of
partial differential equations subject to generic boundary conditions
\begin{align*}
& \mathcal{P} v = 0, && \text{in $\Omega$}, \\
& \mathcal{B} v = 0, && \text{at $\partial \Omega$}.
\end{align*}
It is typically easy to implement a numerical algorithm to solve
the nonhomogeneous version of this set of equations
\begin{align*}
& \mathcal{P} v = \mathcal{F}, && \text{in $\Omega$}, \\
& \mathcal{B} v = \mathcal{J}, && \text{at $\partial \Omega$}.
\end{align*}
To test an implementation we began with the ``manufactured solution,''
$\tilde{v}$, and set
\begin{equation*}
\mathcal{F}_v := \mathcal{P} \tilde{v},
\quad
\mathcal{J}_v := \mathcal{J} \tilde{v}.
\end{equation*}
Thus, given the pair $\{ \mathcal{F}_v, \mathcal{J}_v \}$ we had an exact
solution of the nonhomogeneous problem, namely $\tilde{v}$. While this
does not prove an implementation to be correct, if the function $\tilde{v}$
is chosen to imitate the behavior of anticipated solutions (e.g.,
satisfying the boundary conditions exactly) then this gives us
confidence in our algorithm. 