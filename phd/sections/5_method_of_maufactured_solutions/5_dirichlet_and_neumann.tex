\section{Taylor Series for \texorpdfstring{$\mathcal{E}^{q,p}(x;\varepsilon,\delta)$}{Dirichlet and Neumann Data}}
\label{intro:dirichlet_upper}
Returning to our joint expansion
\bes
\mathcal{E}^{q,p}(x;\varepsilon,\delta) = \sumn \sumn \mathcal{E}_{n,m}^{q,p}(x)\varepsilon^n\delta^m,
\ees
we first calculated the Dirichlet data, $(5.5\text{a})$, when $n=0$. We have
$$\mathcal{E}^{q,p}(x;0,\delta)=\operatorname{exp}\{\pm 0\}=1,$$
therefore 
$$\mathcal{E}^{q,p}_{0,m}(x)= 
\begin{cases} 
1, & m=0, \\
0, & m>0,
\end{cases} $$
and 
$$\mathcal\xi^{u}_{0,m}= 
\begin{cases} 
A_pe^{i\tilde{p}x}, & m=0, \\
0, & m>0,
\end{cases},\quad
\mathcal\xi^{w}_{0,m}= 
\begin{cases} 
B_pe^{i\tilde{p}x}, & m=0, \\
0, & m>0.
\end{cases}
$$
We then evaluated $(5.5\text{a})$ when $n>0$. Following the technique of Pourahmadi \cite{Pourahmadi84} (and of Marchant and Roberts \cite{Roberts83,MarchantRoberts87}), we observed that
\be
\partial_{\varepsilon}\mathcal{E}^{q,p}(x;\varepsilon,\delta)=\left(\pm i\gamma_p^{u}(\delta)f(x)\right)\mathcal{E}^{q,p}(x;\varepsilon,\delta).
\ee
Inserting the Taylor series expansions for $\mathcal{E}^{q,p}$ and $\gamma_p^{q}$ gives
$$\sum_{n=1}^{\infty}\sum_{m=0}^{\infty}\mathcal{E}^{q,p}_{n,m}n\varepsilon^{n-1}\delta^m =
(\pm if)\left(\sum_{r=0}^{\infty}\gamma^{q}_{p,r}\delta^r\right)\left(\sum_{n=0}^{\infty}\sum_{m=0}^{\infty}\mathcal{E}^{q,p}_{n,m}\varepsilon^n\delta^m\right).$$
Re--indexing the left--hand side and rearranging the order of terms on the right--hand side forms
$$ \sum_{n=0}^{\infty}\sum_{m=0}^{\infty}\mathcal{E}^{q,p}_{n+1,m}(n+1)\varepsilon^{n}\delta^m =
 \sum_{n=0}^{\infty}\sum_{m=0}^{\infty}\left((\pm if)\sum_{r=0}^m \gamma^{q}_{p,m-r}\mathcal{E}^{q,p}_{n,r}\right)\varepsilon^n\delta^m.$$
Upon equating like orders we found
$$\mathcal{E}^{q,p}_{n+1,m} = \pm \frac{if}{n+1}\sum_{r=0}^m \gamma^{q}_{p,m-r}\mathcal{E}^{q,p}_{n,r}.$$
Therefore we have
\be
\begin{aligned}
\mathcal{\xi}^u_{n+1,m} &= A_pe^{i\tilde{p}x}\frac{if}{n+1}\sum_{r=0}^m \gamma^{u}_{p,m-r}\mathcal{E}^{u,p}_{n,r}, \\
\mathcal{\xi}^w_{n+1,m} &= -B_pe^{i\tilde{p}x}\frac{if}{n+1}\sum_{r=0}^m \gamma^{w}_{p,m-r}\mathcal{E}^{w,p}_{n,r},
\end{aligned}
\ee
where the initial data is
\be
\mathcal\xi^{u}_{0,m}= 
\begin{cases} 
A_pe^{i\tilde{p}x}, & m=0, \\
0, & m>0,
\end{cases},\quad
\mathcal\xi^{w}_{0,m}= 
\begin{cases} 
B_pe^{i\tilde{p}x}, & m=0, \\
0, & m>0.
\end{cases}
\ee
As $(5.14)$ and $(5.15)$ are valid for all values of $m$, we see that to find the coefficient at order $(n+1,m),$ one only needs the values of $(n,0),\ldots ,(n,m)$. As an example, we have $\xi^q_{0,m}$ from $(5.15)$ which can be used to obtain $\xi^q_{1,m}$ by $(5.14)$. We can then recover all of the $\xi^q_{n,m}$.
\\
\newline
We then calculated the Neumann data, $(5.5\text{b})$, when $n=0$. We have
$$\mathcal{\nu}^u_{0,m}= 
\begin{cases} 
-i\gamma^{u}_{p,0}\xi_{0,0}^u, & m=0, \\
-i\gamma^{u}_{p,m}\xi_{0,m}^u, & m>0,
\end{cases},\quad
\mathcal{\nu}^w_{0,m}= 
\begin{cases} 
i\gamma^{w}_{p,0}\xi_{0,0}^w, & m=0, \\
i\gamma^{w}_{p,m}\xi_{0,m}^w, & m>0,
\end{cases} $$
therefore
$$\mathcal{\nu}^u_{0,m}= 
\begin{cases} 
-i\gamma^{u}_{p,0}A_pe^{i\tilde{p}x}, & m=0, \\
0, & m>0,
\end{cases} \quad 
\mathcal{\nu}^w_{0,m}= 
\begin{cases} 
i\gamma^{w}_{p,0}B_pe^{i\tilde{p}x}, & m=0, \\
0, & m>0.
\end{cases} $$
For $(5.5\text{b})$ and $n>0$ we inserted the Taylor series expansions for $\xi^q$ and $\gamma_p^{q}$ and used $(5.13)$ to deduce
\begin{align*}\sum_{n=1}^{\infty}\sum_{m=0}^{\infty}\nu^q_{n,m}n\varepsilon^{n-1}\delta^m &= 
f\left(\sum_{r=0}^{\infty}\gamma^{q}_{p,r}\delta^r\right)\left(\sum_{k=0}^{\infty}\gamma^{q}_{p,k}\delta^k\right)\left(\sum_{n=0}^{\infty}\sum_{m=0}^{\infty}\xi^q_{n,m}\varepsilon^n\delta^m\right)\\&\mp
(\tilde{p}ff_x)\left(\sum_{r=0}^{\infty}\gamma^{q}_{p,r}\delta^r\right)\left(\sum_{n=1}^{\infty}\sum_{m=0}^{\infty}\xi^q_{n-1,m}\varepsilon^n\delta^m\right).\end{align*}
Re--indexing the left--hand side and rearranging the order of terms on the right--hand side forms
\begin{align*}\sum_{n=0}^{\infty}\sum_{m=0}^{\infty}\nu^q_{n+1,m}(n+1)\varepsilon^n\delta^m&=
\sum_{n=0}^{\infty}\sum_{m=0}^{\infty}\left(f\sum_{r=0}^m \sum_{k=0}^r \gamma^{q}_{p,m-r}\gamma^{q}_{p,r-k}\xi^q_{n,k}\right)\varepsilon^n\delta^m\\&\mp
\sum_{n=1}^{\infty}\sum_{m=0}^{\infty}\left((\tilde{p}ff_x)\sum_{r=0}^m  \gamma^{q}_{p,m-r}\xi^q_{n-1,r}\right)\varepsilon^n\delta^m
.\end{align*}
Upon equating like orders we found
\begin{align}
\begin{split}
\nu^u_{n+1,m} &= \frac{f}{n+1}\sum_{r=0}^m \sum_{k=0}^r \gamma^{u}_{p,m-r}\gamma^{u}_{p,r-k}\xi^u_{n,k}-
\frac{\tilde{p}ff_x}{n+1}\sum_{r=0}^m  \gamma^{u}_{p,m-r}\xi^u_{n-1,r},\\
\nu^w_{n+1,m} &= \frac{f}{n+1}\sum_{r=0}^m \sum_{k=0}^r \gamma^{w}_{p,m-r}\gamma^{w}_{p,r-k}\xi^w_{n,k}+
\frac{\tilde{p}ff_x}{n+1}\sum_{r=0}^m  \gamma^{w}_{p,m-r}\xi^w_{n-1,r},
\end{split}
\end{align}
where our initial data is
\begin{equation}\mathcal{\nu}^u_{0,m}= 
\begin{cases} 
-i\gamma^{u}_{p,0}A_pe^{i\tilde{p}x}, & m=0, \\
0, & m>0,
\end{cases}  
 \quad 
\mathcal{\nu}^w_{0,m}= 
\begin{cases} 
i\gamma^{w}_{p,0}B_pe^{i\tilde{p}x}, & m=0, \\
0, & m>0.
\end{cases} \end{equation}
Analogously to the Dirichlet data, we see that $(5.16)$ and $(5.17)$ are valid for all values of $m$. Therefore we can recover the coefficient at order $(n+1,m)$ by the values of the coefficients at order $(n,0),\ldots ,(n,m)$.
\\
\newline
Finally, we calculated the surface data, $(5.7\text{a})$, when $n=0$. We have
$$\mathcal{E}^{u,p}(x;0,\delta)=\operatorname{exp}\{-0\}=1,$$
therefore 
$$\mathcal{E}^{u,p}_{0,m}(x)= 
\begin{cases} 
1, & m=0, \\
0, & m>0,
\end{cases} $$
and 
$$\zeta_{0,m}= 
\begin{cases} 
-1, & m=0, \\
0, & m>0.
\end{cases}
$$
We then evaluated $(5.7\text{a})$ when $n>0$. 
Inserting the Taylor series expansions for $\mathcal{E}^{u,p}$ and $\gamma_p^{u}$ and applying $(5.13)$ gives
$$\sum_{n=1}^{\infty}\sum_{m=0}^{\infty}\mathcal{E}^{u,p}_{n,m}n\varepsilon^{n-1}\delta^m =
(-if)\left(\sum_{r=0}^{\infty}\gamma^{u}_{p,r}\delta^r\right)\left(\sum_{n=0}^{\infty}\sum_{m=0}^{\infty}\mathcal{E}^{u,p}_{n,m}\varepsilon^n\delta^m\right).$$
Re--indexing the left--hand side and rearranging the order of terms on the right--hand side forms
$$ \sum_{n=0}^{\infty}\sum_{m=0}^{\infty}\mathcal{E}^{u,p}_{n+1,m}(n+1)\varepsilon^{n}\delta^m =
 \sum_{n=0}^{\infty}\sum_{m=0}^{\infty}\left((-if)\sum_{r=0}^m \gamma^{u}_{p,m-r}\mathcal{E}^{u,p}_{n,r}\right)\varepsilon^n\delta^m.$$
Upon equating like orders we found
$$\mathcal{E}^{u,p}_{n+1,m} = -\frac{if}{n+1}\sum_{r=0}^m \gamma^{u}_{p,m-r}\mathcal{E}^{u,p}_{n,r}.$$
Therefore we have
\be
\zeta_{n+1,m} = \frac{if}{n+1}\sum_{r=0}^m \gamma^{u}_{p,m-r}\mathcal{E}^{u,p}_{n,r}, 
\ee
where the initial data is
\be
\zeta_{0,m}= 
\begin{cases} 
-1, & m=0, \\
0, & m>0.
\end{cases}
\ee
We then evaluated $(5.7\text{b})$ when $n=0$. We have
$$\psi_{0,m}= 
\begin{cases} 
i\gamma^{u}_{p,0}\mathcal{E}^{u,p}_{0,m}, & m=0, \\
i\gamma^{u}_{p,m}\mathcal{E}^{u,p}_{0,m}, & m>0,
\end{cases}$$
therefore
$$\psi_{0,m}= 
\begin{cases} 
i\gamma^{u}_{p,0}, & m=0, \\
0, & m>0.
\end{cases}  $$
For $(5.7\text{b})$ and $n>0$ we expanded
\bes
\alpha = \alpha(\delta)=\sum_{k=0}^{\infty}\alpha_k \delta^k,
\ees
and inserted the Taylor series expansions for $\alpha, \xi^q$, and $\gamma_p^{q}$ and used $(5.13)$ to deduce

\begin{align*}\sum_{n=1}^{\infty}\sum_{m=0}^{\infty}\psi_{n,m}n\varepsilon^{n-1}\delta^m &= 
f\left(\sum_{r=0}^{\infty}\gamma^u_{p,r}\delta^r\right)\left(\sum_{k=0}^{\infty}\gamma^u_{p,k}\delta^k\right)\left(\sum_{n=0}^{\infty}\sum_{m=0}^{\infty}\mathcal{E}^u_{n,m}\varepsilon^n\delta^m\right)\\&+
ff_x\left(\sum_{r=0}^{\infty}\gamma^u_{p,r}\delta^r\right)\left(\sum_{k=0}^{\infty}\alpha_k \delta^k\right)\left(\sum_{n=1}^{\infty}\sum_{m=0}^{\infty}\mathcal{E}^u_{n-1,m}\varepsilon^n\delta^m\right).\end{align*}
Re--indexing the left--hand side and rearranging the order of terms on the right--hand side forms

\begin{align*}\sum_{n=0}^{\infty}\sum_{m=0}^{\infty}\psi_{n+1,m}(n+1)\varepsilon^n\delta^m&=
\sum_{n=0}^{\infty}\sum_{m=0}^{\infty}\left((f)\sum_{r=0}^m \sum_{k=0}^r \gamma^u_{p,m-r}\gamma^u_{p,r-k}\mathcal{E}^u_{n,k}\right)\varepsilon^n\delta^m\\&+
\sum_{n=1}^{\infty}\sum_{m=0}^{\infty}\left(( ff_x)\sum_{r=0}^m \sum_{k=0}^r \gamma^u_{p,m-r}\alpha_{r-k}\mathcal{E}^u_{n-1,k}\right)\varepsilon^n\delta^m
.\end{align*}
Upon equating like orders we found
\begin{align}
\begin{split}
\psi_{n+1,m} &= \frac{f}{n+1}\sum_{r=0}^m \sum_{k=0}^r \gamma^u_{p,m-r}\gamma^u_{p,r-k}\mathcal{E}^u_{n,k} \\&+
\frac{ff_x}{n+1}\sum_{r=0}^m \sum_{k=0}^r \gamma^u_{p,m-r}\alpha_{r-k}\mathcal{E}^u_{n-1,k},
\end{split}
\end{align}
where the initial data is
\begin{equation}\mathcal{\psi}_{0,m}= 
\begin{cases} 
i\gamma^u_{p,0}, & m=0, \\
0, & m>0.
\end{cases}  \end{equation}
As before, we can find the coefficient at order $(n+1,m)$ by the values of the coefficients at $(n,0),\ldots ,(n,m)$.