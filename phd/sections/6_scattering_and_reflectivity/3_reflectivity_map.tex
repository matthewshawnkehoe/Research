\section{The Reflectivity Map}
\label{Sec: Reflectivity Map}

Recalling the solution $(4.3)$ to the Helmholtz equation in the upper layer
$$u(x,z) = \sump a_p e^{i \tilde{p} x + i \gamma^u_p z},$$
we note the
very different character of the solution for wavenumbers $p$
in the set
\bes
\mathcal{U}^u := \left\{ p \in \mathbb{Z}\ |\ \alpha_p^2 < 
  (k^u)^2 \right\},
\ees
and those that are not. From our choice of the branch of the square
root, components of $u(x,z)$ corresponding to $p \in \mathcal{U}^u$
propagate away from the layer interface, while those not in this set
decay exponentially from $z=g(x)$. The latter are called evanescent
waves while the former are propagating (defining the set of propagating
modes $\mathcal{U}^u$) and carry energy away from
the grating. With this in mind we define the efficiencies \cite{Petit80}
\bes
e^u_p := (\gamma^u_p/\gamma^u) \Abs{a_p}^2,
\quad
p \in \mathcal{U}^u,
\ees
and the Reflectivity Map
\vspace{4mm}
\be
\label{Eqn:R}
R := \sum_{p \in \mathcal{U}^u} e^u_p.
\ee
Similar quantities can be defined in the lower layer \cite{Petit80}, and
with these the principle of conservation of energy can be stated for
structures composed entirely of dielectrics
\bes
\sum_{p \in \mathcal{U}^u} e^u_p + \tau^2 \sum_{p \in \mathcal{U}^w} e^w_p = 1.
\ees
In this situation a useful diagnostic of convergence for a numerical scheme
(which we will utilize in our simulations) is the Energy Defect
\be
\label{Eqn:EnergyDefect}
D := 1 - \sum_{p \in \mathcal{U}^u} e^u_p - \tau^2 \sum_{p \in \mathcal{U}^w} e^w_p,
\ee
which should be zero for a purely dielectric structure.
