\vspace{-18mm}
\section{Poles and Residues}
\label{Sec: Poles and Residues}

We conclude our discussion of numerics by considering how the Taylor series in 
$(\Eps,\delta)$ are summed and how they can be used to find poles and residues. As an example, we saw in $\S 2.12$ that the DNO,
$G$, can be approximated through the Fourier--Chebyshev approach by
\bes
G(x;\Eps,\delta) \approx G^{N,M}(x;\Eps,\delta) 
  := \sum_{n=0}^{N} \sum_{m=0}^{M} G_{n,m}(x) \Eps^n \delta^m,
\ees
where
\bes
\label{Eqn:G:NMNx}
G_{n,m}(x) \approx G_{n,m}^{N_x}(x) 
  := \sum_{p=-N_x/2}^{N_x/2-1} \hat{G}_{n,m,p} e^{i \tilde{p} x},
\ees
which, in turn, we can approximate the $\hat{G}_p(\Eps,\delta)$ by
\bes
\hat{G}_p^{N,M}(\Eps,\delta) 
  := \sum_{n=0}^{N} \sum_{m=0}^{M} \hat{G}_{n,m,p} \Eps^n \delta^m.
\ees
The technique of Pad\'e approximation 
\cite{BakerGravesMorris96} has been used with HOPS methods
to great advantage in the past \cite{BrunoReitich93b,NichollsReitich00b}
and we advocate its use here. Classically, this approach seeks to estimate
the truncated Taylor series of a single variable
\bes
Q^N(\rho) := \sum_{n=0}^{N} Q_n \rho^n \approx Q(\rho),
\ees
by the rational function
\bes
[L/M](\rho) := \frac{a^L(\rho)}{b^M(\rho)}
  = \frac{\sum_{\ell=0}^{L} a_{\ell} \rho^{\ell}}
  {1 + \sum_{m=1}^{M} b_m \rho^m},
\quad
L+M=N,
\ees
and
\bes
[L/M](\rho) = Q^N(\rho) + \BigOh{\rho^{L+M+1}},
\ees
where well--known formulas for the coefficients $\{a_{\ell},b_m\}$
can be found in \cite{BakerGravesMorris96}.
Pad\'e approximation enjoys greatly enhanced convergence properties
and we refer the interested reader to $\S2.2$ of Baker \& Graves--Morris
\cite{BakerGravesMorris96} and the insightful calculations
of $\S8.3$ of Bender \& Orszag \cite{BenderOrszag78} for a
thorough discussion of the capabilities and limitations of
Pad\'e approximants.

In the current context of functions analytic with respect to two
perturbation variables we utilize the polar coordinates
\bes
\Eps = \rho \cos(\theta),
\quad
\delta = \rho \sin(\theta),
\ees
and write the function
\begin{align*}
\hat{G}_p(\Eps,\delta) 
  & = \sumn \summ \hat{G}_{n,m,p} \Eps^n \delta^m \\
  & = \sumn \summ \left( \hat{G}_{n,m,p} \cos^n(\theta) \sin^m(\theta) \right) \rho^{n+m}.
\end{align*}
Letting $\ell = n + m$ and $s = m$ we can write this as
\bes
\hat{G}_p(\Eps,\delta) = \sum_{\ell=0}^{\infty} \left\{ \sum_{s=0}^{\ell} 
    \hat{G}_{\ell-s,s,p} \cos^{\ell-s}(\theta) \sin^s(\theta) \right\}
    \rho^{\ell}
  =: \sum_{\ell=0}^{\infty} \tilde{G}_{\ell,p}(\theta) \rho^{\ell}.
\ees
We then select particular values of $\theta = \theta_j$ between $0$ and $2 \pi$
and apply classical Pad\'e approximation on the resulting
$\{ \tilde{G}_{\ell,p}(\theta_j) \}$ as a function of $\rho$ alone.