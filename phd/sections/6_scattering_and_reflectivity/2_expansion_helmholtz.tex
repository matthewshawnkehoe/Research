\section{Expansion of Surface Data}
\label{Sec: Expansion of Helmholtz}
In $\S 4.2$ we saw that the surface data for our two--layer scattering problem is given by
\begin{subequations}
\begin{align}
\zeta(x) & = -e^{-i \gamma^u g(x)}, 
  \label{Eqn:Govern:e} \\
\psi(x) & = (i \gamma^u + i \alpha (\partial_x g)) e^{-i \gamma^u g(x)}.
  \label{Eqn:Govern:f}
\end{align}
\end{subequations}
Following the strategy discussed in Chapter 5, we selected a wavenumber $p$ and defined
\bes
\mathcal{E}^{u,p}(x;\varepsilon,\delta):=\operatorname{exp}\left\{-i\gamma_p^{u}(\delta)\varepsilon f(x)\right\},
\ees
and then derived the terms $\mathcal{E}_{n,m}^{u,p}$ in the expansion
\bes
\mathcal{E}^{u,p}(x;\varepsilon,\delta) = \sumn \sumn \mathcal{E}_{n,m}^{u,p}(x)\varepsilon^n\delta^m.
\ees
From this, it is clear that
\begin{subequations}
\begin{align}
\zeta_{n,m}(x) &= -\mathcal{E}_{n,m}^{u,p}(x),\\
\psi_{n,m}(x) &= \left(i\gamma_p^{u} + i\alpha\varepsilon f_x(x)\right)\mathcal{E}_{n,m}^{u,p}(x).
\end{align}
\end{subequations}
We first calculated the Dirichlet data, $(6.2\text{a})$, when $n=0$. We have
$$\mathcal{E}^{u,p}(x;0,\delta)=\operatorname{exp}\{-0\}=1,$$
therefore 
$$\mathcal{E}^{u,p}_{0,m}(x)= 
\begin{cases} 
1, & m=0, \\
0, & m>0,
\end{cases} $$
and 
$$\zeta_{0,m}= 
\begin{cases} 
-1, & m=0, \\
0, & m>0.
\end{cases}
$$
We then evaluated $(6.2\text{a})$ when $n>0$. 
Inserting the Taylor series expansions for $\mathcal{E}^{u,p}$ and $\gamma_p^{u}$ and applying $(5.11)$ gives
$$\sum_{n=1}^{\infty}\sum_{m=0}^{\infty}\mathcal{E}^{u,p}_{n,m}n\varepsilon^{n-1}\delta^m =
(-if)\left(\sum_{r=0}^{\infty}\gamma^{u}_{p,r}\delta^r\right)\left(\sum_{n=0}^{\infty}\sum_{m=0}^{\infty}\mathcal{E}^{u,p}_{n,m}\varepsilon^n\delta^m\right).$$
Re--indexing the left--hand side and rearranging the order of terms on the right--hand side forms
$$ \sum_{n=0}^{\infty}\sum_{m=0}^{\infty}\mathcal{E}^{u,p}_{n+1,m}(n+1)\varepsilon^{n}\delta^m =
 \sum_{n=0}^{\infty}\sum_{m=0}^{\infty}\left((-if)\sum_{r=0}^m \gamma^{u}_{p,m-r}\mathcal{E}^{u,p}_{n,r}\right)\varepsilon^n\delta^m.$$
Upon equating like orders we found
$$\mathcal{E}^{u,p}_{n+1,m} = -\frac{if}{n+1}\sum_{r=0}^m \gamma^{u}_{p,m-r}\mathcal{E}^{u,p}_{n,r}.$$
Therefore we have
\be
\zeta_{n+1,m} = \frac{if}{n+1}\sum_{r=0}^m \gamma^{u}_{p,m-r}\mathcal{E}^{u,p}_{n,r}, 
\ee
where the initial data is
\be
\zeta_{0,m}= 
\begin{cases} 
-1, & m=0, \\
0, & m>0.
\end{cases}
\ee
As $(6.3)$ and $(6.4)$ are valid for all values of $m$, we see that to find the coefficient at order $(n+1,m),$ one only needs the values of $(n,0),\ldots ,(n,m)$.
\\
\newline
We then calculated $(6.2\text{b})$ when $n=0$. We have
$$\psi_{0,m}= 
\begin{cases} 
i\gamma^{u}_{p,0}\mathcal{E}^{u,p}_{0,m}, & m=0, \\
i\gamma^{u}_{p,m}\mathcal{E}^{u,p}_{0,m}, & m>0,
\end{cases}$$
therefore
$$\psi_{0,m}= 
\begin{cases} 
i\gamma^{u}_{p,0}, & m=0, \\
0, & m>0.
\end{cases}  $$
For $(6.2\text{b})$ and $n>0$ we expanded
\bes
\alpha = \alpha(\delta)=\sum_{k=0}^{\infty}\alpha_k \delta^k,
\ees
and inserted the Taylor series expansions for $\alpha, \xi^q$, and $\gamma_p^{q}$ and used $(5.11)$ to deduce

\begin{align*}\sum_{n=1}^{\infty}\sum_{m=0}^{\infty}\psi_{n,m}n\varepsilon^{n-1}\delta^m &= 
f\left(\sum_{r=0}^{\infty}\gamma^u_{p,r}\delta^r\right)\left(\sum_{k=0}^{\infty}\gamma^u_{p,k}\delta^k\right)\left(\sum_{n=0}^{\infty}\sum_{m=0}^{\infty}\mathcal{E}^u_{n,m}\varepsilon^n\delta^m\right)\\&+
ff_x\left(\sum_{r=0}^{\infty}\gamma^u_{p,r}\delta^r\right)\left(\sum_{k=0}^{\infty}\alpha_k \delta^k\right)\left(\sum_{n=1}^{\infty}\sum_{m=0}^{\infty}\mathcal{E}^u_{n-1,m}\varepsilon^n\delta^m\right).\end{align*}
Re--indexing the left--hand side and rearranging the order of terms on the right--hand side forms

\begin{align*}\sum_{n=0}^{\infty}\sum_{m=0}^{\infty}\psi_{n+1,m}(n+1)\varepsilon^n\delta^m&=
\sum_{n=0}^{\infty}\sum_{m=0}^{\infty}\left((f)\sum_{r=0}^m \sum_{k=0}^r \gamma^u_{p,m-r}\gamma^u_{p,r-k}\mathcal{E}^u_{n,k}\right)\varepsilon^n\delta^m\\&+
\sum_{n=1}^{\infty}\sum_{m=0}^{\infty}\left(( ff_x)\sum_{r=0}^m \sum_{k=0}^r \gamma^u_{p,m-r}\alpha_{r-k}\mathcal{E}^u_{n-1,k}\right)\varepsilon^n\delta^m
.\end{align*}
Upon equating like orders we found
\begin{align}
\begin{split}
\psi_{n+1,m} &= \frac{f}{n+1}\sum_{r=0}^m \sum_{k=0}^r \gamma^u_{p,m-r}\gamma^u_{p,r-k}\mathcal{E}^u_{n,k} \\&+
\frac{ff_x}{n+1}\sum_{r=0}^m \sum_{k=0}^r \gamma^u_{p,m-r}\alpha_{r-k}\mathcal{E}^u_{n-1,k},
\end{split}
\end{align}
where the initial data is
\begin{equation}\mathcal{\psi}_{0,m}= 
\begin{cases} 
i\gamma^u_{p,0}, & m=0, \\
0, & m>0.
\end{cases}  \end{equation}
Analogously to the Dirichlet data, we see that $(6.5)$ and $(6.6)$ are valid for all values of $m$. Therefore we can find the coefficient at order $(n+1,m)$ by the values of the coefficients at $(n,0),\ldots ,(n,m)$.