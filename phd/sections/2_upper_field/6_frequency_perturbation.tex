\section{Frequency Perturbation}
\label{intro:frequency_perturbation}
We now perform an Asymptotic Waveform Evaluation by writing the illumination frequency as 
\begin{equation}\omega=(1+\delta)\underline{\omega}=\underline{\omega}+\delta\underline{\omega},~\delta\in\mathbb R, ~ \delta \ll 1.\end{equation}
With this we see that
\begin{subequations}
\begin{align} 
k^u & = \omega/c^u = (1 + \delta) \uomega/c^u 
  =: (1 + \delta) \uk^u = \uk^u + \delta \uk^u,\\
\alpha & = k^u \sin(\theta) = (1 + \delta) \uk^u \sin(\theta)
  =: (1 + \delta) \ualpha = \ualpha + \delta \ualpha,\\
\gamma^u & = k^u \cos(\theta) = (1 + \delta) \uk^u \cos(\theta)
  =: (1 + \delta) \ugamma^u = \ugamma^u + \delta \ugamma^u.
\end{align}
\end{subequations}
We can relate the constants in the underscore variables by the relationship
\begin{align}\underline{\alpha}^2 + (\underline{\gamma}^u)^2 = (\underline{k}^u)^2.
\end{align}
Then, since $({\gamma}^u)^2=(\delta+1)^2\left((\underline{k}^u)^2-\underline{\alpha}^2\right)=(\delta+1)^2(\underline{\gamma}^u)^2$, $(2.18)$ becomes
\begin{align*}F\left(x,z;f,u,\underline{\alpha},\underline{\gamma}^u\right)&=-\text{div}[A_1\nabla u]-\text{div}[A_2\nabla u]-B_1\nabla u - B_2\nabla u\nonumber \\&~~-2S_1i\underline{\alpha}\partial_xu-2S_1i\underline{\alpha}\delta\partial_xu-S_1\delta^2(\underline{\gamma}^u)^2u-2S_1\delta(\underline{\gamma}^u)^2u-S_1(\underline{\gamma}^u)^2u\nonumber
\\&~~-2S_2i\underline{\alpha}\partial_xu-2S_2i\underline{\alpha}\delta\partial_xu-S_2\delta^2(\underline{\gamma}^u)^2u-2S_2\delta(\underline{\gamma}^u)^2u-S_2(\underline{\gamma}^u)^2u.
\end{align*}
Also, the left-hand side of $(2.20\text{a})$ becomes
\begin{equation*}\Delta u +2i\underline{\alpha}\partial_xu+2i\underline{\alpha}\delta\partial_xu+\delta^2(\underline{\gamma}^u)^2u+2\delta(\underline{\gamma}^u)^2u+(\underline{\gamma}^u)^2u,\end{equation*}
and the boundary condition for $(2.20\text{a})$ becomes
\begin{equation}\Delta u +2i\underline{\alpha}\partial_{x}u+(\underline{\gamma}^u)^2u=\tilde{F}\left(x,z;f,u,\underline{\alpha},\underline{\gamma}^u\right),\quad\text{$0<z<a$}. \end{equation}
We move all terms with $\delta$ to the right--hand side to form
\begin{align*}\tilde{F}\left(x,z;f,u,\underline{\alpha},\underline{\gamma}^u\right)&=-\text{div}[A_1\nabla u]-\text{div}[A_2\nabla u]-B_1\nabla u - B_2\nabla u\nonumber
\\&~~-2i\underline{\alpha}\delta\partial_xu-\delta^2(\underline{\gamma}^u)^2u-2\delta(\underline{\gamma}^u)^2u\nonumber
\\&~~-2S_1i\underline{\alpha}\partial_xu-2S_1i\underline{\alpha}\delta\partial_xu-S_1\delta^2(\underline{\gamma}^u)^2u-2S_1\delta(\underline{\gamma}^u)^2u-S_1(\underline{\gamma}^u)^2u\nonumber
\\&~~-2S_2i\underline{\alpha}\partial_xu-2S_2i\underline{\alpha}\delta\partial_xu-S_2\delta^2(\underline{\gamma}^u)^2u-2S_2\delta(\underline{\gamma}^u)^2u-S_2(\underline{\gamma}^u)^2u.
\end{align*}
The boundary condition $(2.20\text{d})$ becomes
$$\partial_z \left[u(x,a)\right] - T_0^u[u(x,a)]=\tilde{P}(x;f,u),$$
where $T_0^u = i \ugamma_D^u$
corresponds to the case where $\delta=0$ and
\bes
\tilde{P}(x;f,u) = -\frac{1}{a} (\Eps f(x)) T^u \left[ u(x,a) \right]
+ (T^u-T_0^u) \left[ u(x,a) \right].
\ees
Our governing equations are now
\begin{subequations}
\begin{align}
\Delta u +2i\underline{\alpha}\partial_{x}u+(\underline{\gamma}^u)^2u&=\tilde{F}\left(x,z;g,u,\underline{\alpha},\underline{\gamma}^u\right),  &&\text{$0<z<a$}, \\
u(x,0)&=\zeta^u(x),&& \text{at $z=0$},\\
u(x+d,z)&=u(x,z),\\
\partial_z \left[u(x,a)\right] - T_0^u[u(x,a)]&=\tilde{P}(x;f,u),&& \text{at $z=a$}.
\end{align}
\end{subequations}
