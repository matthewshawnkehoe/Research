\section{Sobolev Spaces and Elliptic Theory}
\label{intro:sobolev_spaces_elliptic_theory}
We summarize the characterization of the Sobolev spaces $H^s = W^{s,2}$ applied in laterally $d$--periodic functions relevant to scattering problems of interest to us. We know that any $d$--periodic $L^2$ function
$$\mu(x+d)=\mu(x),$$
can be expressed as
$$\mu(x) = \sum_{p=-\infty}^{\infty}\hat{\mu}_p e^{i\tilde{p} x},$$
where 
$$\tilde{p}:= \left(\frac{2\pi}{d}\right)p,\quad \hat{\mu}_p = \frac{1}{d}\int_0^d \mu(x) e^{-i\tilde{p} x}\diff x.$$
We then define our $x$--periodic norms. For any $L^2$ function $\mu=\mu(x)$, we recall the classical Sobolev norm for any real $s\geq 0$:
$$\| \mu\|_{H_x^s}^2:=\sum_{p=-\infty}^{\infty}\langle \tilde{p} \rangle^{2s}|\hat{\mu}_p|^2,\quad \langle \tilde{p} \rangle^2 := 1 + |\tilde{p}|^2.$$
For the $L^2$ function $u=u(x,z)$ we require the classical Sobolev norm for any integer $s\geq 0$ and $a > 0$
\bes
\| u\|_{H_{x,z}^s}^2 := \sum_{\ell=0}^{s} \sump \Angle{\tilde{p}}^{2(s-\ell)}
  \int_0^a \Abs{\hat{u}_p(z)}^2  \diff z
  = \sum_{\ell=0}^{s} \sump \Angle{\tilde{p}}^{2(s-\ell)}
  \Norm{\hat{u}_p}{L^2([0,a])}^2.
\ees
With these norms, we define the following function spaces. First, for real $s\geq 0$,
\bes
H^s\big([0,d]\big):=\left\{\mu(x)\in L^2\big([0,d]\big)~\big|~\|\mu\|_{H_x^s}< \infty\right\}.
\ees
Also, for any integer $s\geq 0$,
\bes
H^s\big([0,d]\times[0,a]\big):=\left\{u(x)\in L^2\big([0,d]\times[0,a]\big)~\big|~\|u\|_{H_{x,z}^s}< \infty\right\}.
\ees
With these we can now establish the following three properties based on classical elliptic theory. The first property is the ``Algebra Property" of Sobolev spaces which allows us to estimate products of functions in our function classes. The second property is a theorem which gives a rigorous statement of the ``Elliptic Estimate." The final property provides a method of bounding translated elements in our function spaces.
\vskip 0.1in
\begin{lemma}
Given an integer $s \ge 0$ and any $\sigma > 0$, there exists a constant $\mathcal{M}=\mathcal{M}(s)$ such that if $f\in C^s([0,d]),u\in H^s([0,d]\times [0,a])$ then
\begin{equation}
\|fu\|_{H^s} \le \mathcal{M}|f|_{C^s}\|u\|_{H^s},
\end{equation}
and if $\tilde{f}\in C^{s+1/2+\sigma}([0,d]),\tilde{u}\in H^{s+1/2}([0,d])$ then there exists a constant $\tilde{\mathcal{M}}=\tilde{\mathcal{M}}(s)$ such that
\begin{equation}
 \|\tilde{f}\tilde{u}\|_{H^{s+1/2}} \le \tilde{\mathcal{M}}|\tilde{f}|_{C^{s+1/2+\sigma}}\|\tilde{u}\|_{H^{s+1/2}}.   
\end{equation}
\end{lemma}
\vskip 0.1in
\begin{theorem} Given an integer $s\ge 0$, if $F\in H^s([0,d])\times [0,a]),$ $\zeta^u \in H^{s+3/2}([0,d]),$ $P\in H^{s+1/2}([0,d])$, then there exists a unique solution $u\in H^{s+2}([0,d])\times [0,a])$ of 
\begin{subequations}
\begin{align}
\Delta u(x,z) +2i\underline{\alpha}\partial_{x}u(x,z)+(\underline{\gamma}^u)^2u(x,z)&=F(x,z), && \text{$0<z<a$},\\
u(x,0)&=\zeta^u(x,0), && \text{at $z=0$},\\
u(x+d,z)&=u(x,z),\\
\partial_z u(x,a)-T_0^u[u(x,a)] &= P(x), && \text{at $z=a$},
\end{align}
\end{subequations}
satisfying
\begin{equation}\|u\|_{H^{s+2}}\le C_e\{\|F\|_{H^{s}}+\|\zeta^u\|_{H^{s+3/2}}+\|P\|_{H^{s+1/2}}  \}, \end{equation}
for some constant $C_e = C_e(s) > 0$.
\end{theorem}
\vskip 0.1in
\begin{lemma}
Given an integer $s\ge 0$, if $F\in H^s([0,d])\times [0,a])$, then $(a-z)F \in$ $H^s([0,d])\times [0,a])$ and there exists a positive constant $Z_a = Z_a(s)$ such that
$$\|(a-z)F\|_{H^s} \le Z_a \|F\|_{H^s}.$$
\end{lemma}
\begin{flushleft}
The proof of these three properties is established in Appendix B.
\end{flushleft}