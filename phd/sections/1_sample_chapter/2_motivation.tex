\section{Motivation}
\label{intro:motivation}

The scattering of linear electromagnetic waves by a layered structure is a
central model in many problems of scientific and engineering interest. Examples arise in areas such as geophysics \cite{VirieuxOperto09,BleibRondenay09}, imaging \cite{NW01}, materials science \cite{Godreche92}, nanoplasmonics \cite{Raether88,Maier07,EnochBonod12}, and oceanography \cite{BL82}. In the case of nanoplasmonics, there are many topics of interest such as extraordinary optical transmission \cite{ELGTW98}, surface enhanced spectroscopy \cite{Moskovits85}, and surface plasmon resonance biosensing \cite{Homola08,ILWJLNNO11} and \cite{LJJOO12,JJJLWO13,RJOM13,NichollsReitichJohnsonOh14}. In all of the physical problems it is necessary to approximate scattering returns in a fast, robust, and highly accurate fashion. This thesis will expand upon a novel HOPS algorithm \cite{Nicholls14b,NichollsTammali15,NichollsOhJohnsonReitich15} designed for the numerical simulation of the layered periodic media (diffraction or scattering) problem.

A variety of classical algorithms have been used for simulation of this problem. However, recent studies have demonstrated \cite{AmbroseNicholls13,Nicholls14b,nicholls2016high,NichollsOhJohnsonReitich15} that volumetric approaches (such as finite difference and finite/spectral element methods) are greatly disadvantaged when dealing with layered media problems because of
the large number of unknowns. Another natural candidate is an interfacial method based upon integral equations (IEs) \cite{ColtonKress13}. There are, however, also difficulties associated with these, as discussed in \cite{AmbroseNicholls13,Nicholls14b,nicholls2016high,NichollsOhJohnsonReitich15}. In the past few years, a number of these have been addressed through various techniques such as (i) the use of sophisticated quadrature rules to deliver high order spectral accuracy, (ii) the design of preconditioned iterative solvers with appropriate acceleration \cite{GreengardRokhlin87}, and (iii) new tactics to avoid periodizing the Green function \cite{BarnettGreengard11,ChoBarnett15,LaiKobayashiBarnett15}. Despite these alternatives (see, e.g., \cite{ReitichTamma04}), there are two properties that make these strategies noncompetitive in our parametrized setting.  These are:
\begin{enumerate}
    \item For configurations parameterized by a real value $\varepsilon$ (in our scheme the height/slope of the interface), an IE solver will return the scattering returns for only one particular value of $\varepsilon$. If this is changed, the solver must be run again.
    \item IE solvers require inverting a dense, nonsymmetric positive definite system of linear equations for every simulation.
\end{enumerate}
In contrast, the HOPS approach \cite{Nicholls14b,nicholls2016high,NichollsOhJohnsonReitich15} can effectively address these concerns. More specifically, in \cite{nicholls2016high,NichollsOhJohnsonReitich15} an alternative known as the \gls{fe} method is proposed which is based on the low-order calculations of Rayleigh \cite{Rayleigh07} and Rice \cite{Rice51}. An expansion to high order was first introduced by Bruno and Reitich \cite{BrunoReitich93a,BrunoReitich93b,BrunoReitich93c} and then was later enhanced and stabilized by Nicholls, Reitich, and Malcolm \cite{NichollsReitich03a,NichollsReitich03b,NichollsReitich07,MalcolmNicholls10}. This latter method is known as the TFE method. The TFE method maintains all of the classical advantages of IE formulations (such as surface formulation and exact enforcement of far--field and quasi--periodic boundary conditions) while effectively addressing the two shortcomings listed above:
\begin{enumerate}
    \item The method is built upon expanding in the boundary parameter $\varepsilon$. Once the Taylor coefficients are known for the scattering quantities, the TFE method can recover all of the returns by summing the Taylor coefficients. It is unnecessary to begin a new summation for every value of $\varepsilon$. 
    \item The scheme is based on a perturbation of the interface which, at every perturbation order, requires the inversion of a single, sparse operator corresponding to the flat-interface solution.
\end{enumerate}
For a single incident wavelength, the TFE method is among the most efficient available in our layered media setting. A generalization of the HOPS approach developed by Bruno and Reitich is known as an \gls{awe}. The AWE methods \cite{PillageRohrer90,KSNZA96,ReddyDeshpandeCockrellBeck98,SloaneLeeLee01,Nicholls16} are built upon an additional expansion in wavelength (frequency) about a base value and will be a major source of analysis in the second half of our thesis. Our aim is to develop a novel interfacial method using a combined HOPS/AWE algorithm that provides a stable numerical scheme and a rigorous convergence result. We will carefully show that our new algorithm is highly accurate, rapid, robust, and is jointly analytic with respect to two smallness assumptions: (i) an interfacial deformation and (ii) a frequency deformation.